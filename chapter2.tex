%DO NOT MESS AROUND WITH THE CODE ON THIS PAGE UNLESS YOU %REALLY KNOW WHAT YOU ARE DOING
\chapter{Problem Statement} \label{Problem Statement}
\section{Literature Review} \label{Literature Review}
Before the design phase of the collection mechanism assembly, a little insight into already existing similar mechanisms were looked upon. There was one research paper of an electrically powered cow dung collecting machine.\\
Animal Cow dung Cleaner[1]. This is a portable cow dung cleaning machine which has electrically powered conveyor belt which takes collected cow dung from the ground to the bin at rear. This machine is very heavy as it has batteries and a motor, and needs to be built with heavier materials to sustain it's own weight.\\
Automated Customized Cow Shed Cleaning Machine[2]. This is an entire shed wise cleaning mechanism which cleans cow dung from the shed floor by the means of wipers linked to rails. This system can easily clean an entire area at a very high speed and efficiency. While it is efficient it's very costly to maintain and is not portable.

\section{Problem Definition} \label{Problem Definition}
Cow dung is a viscous, dense and smelly substance. In the current scenario, it is being collected manually which is tedious and time consuming. Cow shelters have to spend 2-3 hours cleaning one shed, and it adds up as the number of sheds increase. Also, cow dung on streets is never cleaned nor collected for the reason being it’s too tedious. Cow dung can be used to generate energy via biogas plants, yet a lot of it is always wasted due to the above mentioned reasons.

\section{Scope of Work} \label{Scope of Work}

\begin{itemize}
    \item Survey the current collection capacity of cow dung in cow shelters.
    \item Design and analyse individual components
     \item Fabricate working model
\end{itemize}


